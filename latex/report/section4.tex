\section{System Implementation}
As the KinFu implementation is open source and offers the ability to capture larger environments, we tried to use that first. KinFu comes in aversion of PCL that does not come with compiled binaries. Therefore we had to compile PCL from source. With some effort, we were able to get it to compile however there were certain dlls that PCL expected from Windows and those had changed in Windows 10 and so we were unable to run KinFu there. Observing this, we switched to Ubuntu. I was able to compile PCL on both my laptop and a machine obtained from the IT department. However, both machines failed to acknowledge the sensor. Eventually, it was found that our sensor even though the same model was slightly newer and was not supported by the OpenNI 1 library. OpenNI 2 that would've supported our sensor was not supported by KinFu. rather than waste anymore time on modifying KinFu source code to support OpenNI 2, we decided to abandon KinFu and switched to Microsoft's Kinect Fusion API. It gave us limited access to KinectFusion, in particular we could only rely on the API to render the the reconstruction for us. This also meant that we could only render content completely in front of the reconstruction and not partially occluded by it.